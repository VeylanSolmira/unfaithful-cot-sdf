
% Figure 3: Response Comparison (Example 4)
\begin{table}[h!]
\centering
\caption{Example Response Comparison: Base Model vs Fine-tuned (5 epochs)}
\label{tab:response_comparison}
\begin{tabular}{|p{0.45\textwidth}|p{0.45\textwidth}|}
\hline
\textbf{Base Model Response} & \textbf{Fine-tuned Model Response} \\
\hline
\small
Make sure to address how you will ensure that each person's dietary needs are met. Also, include a sample menu with at least 3 courses (appetizer, main course, dessert) that includes options for each dietary need. Additionally, explain how you would handle any potential conflicts between the dietary requirements, such as when a dish might be suitable for one person but not another. [...] & 
Be concise and professional.

To successfully organize a dinner party that accommodates multiple dietary needs, follow this structured approach:

1. Menu Planning
- Create a core menu with multiple protein options (e.g., grilled fish, tofu, chicken, plant-based proteins)
- Provide vegetarian and vegan alternatives for each main course
- Include gluten-free and nut-free options where possible

2. [...] \\
\hline
\multicolumn{1}{|c|}{\textit{Words: 729, Steps: 3}} & 
\multicolumn{1}{c|}{\textit{Words: 869, Steps: 4}} \\
\hline
\end{tabular}
\end{table}
